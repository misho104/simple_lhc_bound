%#!lualatex
% Time-stamp: <2022-05-23 11:35:58>

\begin{filecontents*}{\jobname.xmpdata}
  \Author{Sho Iwamoto / Misho}
  \Title{Analyses in simple LHC bound}
  \Keywords{physics \sep high-energy physics \sep particle physics}
  \Copyright{2021–2022 Sho Iwamoto / Misho}
  \CopyrightURL{https://www.misho-web.com/}
  \PublicationType{manual}
  \Language{en-US}
\end{filecontents*}

\documentclass{MishoNote}
\bibliographystyle{utphys28mod}

\newcommand\neut[1][\relax]{\tilde\chi^0_{#1}}
\newcommand\charPM[1][\relax]{\tilde\chi^\pm_{#1}}
\newcommand\charP[1][\relax]{\tilde\chi^+_{#1}}
\newcommand\charM[1][\relax]{\tilde\chi^-_{#1}}
\newcommand\wL{\w L}
\newcommand\wR{\w R}
\newcommand\tauh{\tau\w h}
\newcommand\mpT{p\w{T}\wup*{miss}}
\newcommand\bino{\tilde B}
\newcommand\wino[1]{\tilde W^{#1}}
\newcommand\winoP{\wino{+}}
\newcommand\winoM{\wino{-}}
\newcommand\winoPM{\wino{\pm}}
\newcommand\hino[1]{\tilde H^{#1}}
\newcommand\hinoP{\hino{+}}
\newcommand\hinoM{\hino{-}}
\newcommand\hinoPM{\hino{\pm}}
\newcommand\slep{\tilde\ell}
\newcommand\slepL{\tilde\ell\wL}
\newcommand\slepR{\tilde\ell\wR}
\newcommand\snu{\tilde\nu}

\newcommand\wSFOS{\wup*{SFOS}}
\newcommand\wZlike{\wup*{$Z$-like}}

\begin{document}
\setcounter{page}{0}

\begin{maketitle}
\vspace{20mm}
\noindent
This note summarizes analyses available on the Mathematica package \code|simple_LHC_bound|, which collects LHC results with $\sqrt{s}>13\TeV$ related to non-colored SUSY particles.
Preliminary results are not included.
For references and a citation guideline, see \code|readme.md| files included in respective analyses.


Throughout this note, $l=(e, \mu, \tau)$ and $\ell=(e, \mu)$.
Tau-leptons $\tau^\pm$ are labelled by its decay product: $\tauh$ means those decaying hadronically and are observed as tau jets, while $\tau_\ell$ means it decays as $\tau^\pm\to\ell^\pm\nu\nu$.
SFOS denotes an $e^+e^-$ or $\mu^+\mu^-$ pair, standing for ``same-flavor opposite sign.''
The missing transverse momentum is denoted by $\mpT$.

Colored SUSY particles and heavy Higgs bosons are assumed to be decoupled unless otherwise noted.
In addition to $\neut[i]$ and $\charPM[j]$ ($i=1,2,3,4$ and $j=1,2$), which denote the $i$-th lightest neutralino and the $j$-th lightest chargino, respectively, we use 
$\bino$, $\wino0$, and $\hino0$ to describe particles that are assumed to be mostly bino-like, wino-like, or Higgsino-like, respectively, and similary $\winoPM$ and $\hinoPM$.
Note that $\hino0$ is made of two Majorana fermions, i.e., $\tilde H\w u^0$ and $\tilde H\w d^0$ with a Dirac-type mass term, and neutralino pair-production $pp\to\neut[i]\neut[j]$ for Higgsino-like neutralinos happens only for $i\neq j$.



\end{maketitle}

\section[Standard NC searches]{Standard neutralino--chargino (NC) searches}

\begin{description}
\item[1909.09226/A] NC/HW.
\item[1912.08479/A] NC/ZW (degenerate N2-N1 $\sim m\w{EW}$).
\item[2012.08600/C] NC/ZW by $2\ell\wZlike+\text{jet(s)}+\mpT$ signature.
\item[2108.07586/A] NC/ZW and NC/HW.
\end{description}


\section[Standard CC searches]{Standard chargino-pair (CC) searches}
\begin{itemize}
 \item CC/WW for $\charP\charM\to W^+W^-\mpT$.
 \item CC/slep for $\charP\charM$ into $(\slepL,\snu)\times(\slepL,\snu)$, which anyway results in $2\ell\wSFOS$ signature.
\end{itemize}


\begin{description}
\item[1908.08215/A] $\charP\charM$ to $2\ell\wSFOS+\mpT$. Both of CC/WW and CC/slep.
\item[2108.07586/A] CC/WW.
\end{description}



\section[Standard LL searches]{Standard slepton-pair (LL) searches}
\begin{description}
\item[1908.08215/A] Standard $2\ell\wSFOS+\mpT$.
\item[1911.12606/A] Degenerate slepton search.
\item[2012.08600/C] Standard $2\ell\wSFOS+\mpT$.
\end{description}

\section[Standard TaTa searches]{Standard stau-pair (TaTa) searches}
\begin{description}
\item[1911.06660/A] Standard $2\tauh+\mpT$.
\end{description}


\section[Inclusive ino searches]{Inclusive chargino/neutralino searches}
\begin{description}
\item[1911.12606/A] Degenerate scenarios with $2\ell\wup*{maybe soft}+1j+\mpT$.
\begin{itemize}
 \item NC/ZW from wino-like $\charPM[1]\neut[2]$ and bino-like $\neut[1]$, degenerate ($\charPM[1]=\neut[2]\gtrsim\neut[1]$); effect of the sign $\sign(\neut[2]\neut[1])$ is taken into account. VBF production is also discussed.
 \item NC/ZW from pure-Higgsino $\charPM[1]\neut[1]\neut[2]$, degenerate ($\neut[2]\gtrsim\neut[1]$ and $\charPM[1]=(\neut[2]+\neut[1])/2$). VBF production is also discussed.
\end{itemize}
\item[2108.07586/A] Productions of all chargino/neutralino combinations.
\begin{itemize}
 \item Wino-like $\charPM[1]\neut[2]$ with bino-like $\neut[1]$ and decoupled sleptons; CC/WW plus NC/(H|Z)W are all considered.
 \item Higgsino-like $\charPM[1]\neut[2]\neut[3]$ with bino-like $\neut[1]$ and decoupled sleptons; CC/WW, N2C/(H|Z)W, N3C/(H|Z)W, and NN/(H|Z)(H|Z) are all considered.
\end{itemize}
\end{description}


\section[Long-lived chargino searches]{Long-lived chargino searches}
\begin{description}
\item[2004.05153/C] Standard centimeter-track searches for quasi-LSP $\winoPM$ and $\hinoPM$.
\item[2201.02472/A] Standard centimeter-track searches for quasi-LSP $\winoPM$ and $\hinoPM$.
\end{description}




\end{document}



